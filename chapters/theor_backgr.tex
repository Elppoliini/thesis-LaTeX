% Theoretical background

\chapter{Theoretical Background} \label{ch:theor_backgr}

The project was developed on top of another student's final year project \cite{theseus:gere-zoltan} to automate the tasks that are required in the beginning of each course that uses his implementation.
Their solution provided the \gls{lora} end device, ChirpStack network server and LORIX One Gateway as starting tools.
The following section is a walk-through of the features the web interface of the ChirpStack server provides and how it is used without the implementation of automation this project provides.
The starting point of this walk-through occurs after the server has been set up and configured.

The end devices that the students use are added to an application in the ChirpStack server.
The server has a clear \gls{ui} which is opened to the Applications page when the user logs in as seen in Figure~\ref{fig:ChirpStack_application_list}.

\begin{figure}[ht]
  \centering
  {\includegraphics[width=\textwidth]{illustration/ChirpStack_application_list.png}}
  \caption{List of applications on a ChirpStack network server.}
  \label{fig:ChirpStack_application_list}
\end{figure}

The page layout provides a sidebar that can be hidden from view with an arrow button next to the ChirpStack logo.
The sidebar consists of two sections.
The first section contains four buttons: Network-servers, Gateway-profiles, Organizations, and All users.
The second section includes a drop-down list to select an organization and seven additional buttons: Org. settings, Org. users, Service-profiles, Device-profiles, Gateways, Applications, and Multicast-groups.
Each button corresponds to a specific category or feature.
Clicking on any of these buttons navigates the user to a dedicated page or section displaying a list of elements associated with that category.
Each element in the list is hyperlinked to provide easy access to its details, allowing users to view and modify the information they have previously provided during creation or addition to the list.
Next, moving the layout and features of each button are discussed individually.

\section{Network-servers}
The Network-servers display the added network servers, providing their names and server details in the hostname:port format.
Users can click the Add button to create a new network server to the list.
When creating a new server, users must provide a server name and specify the server details, including the hostname and port in a single field, as mandatory fields.
Additionally, users have the option to include additional features, such as enabling gateway discovery and providing \gls{ca} and \gls{tls} certificates, along with \gls{tls} keys for both ChirpStack Application Server to ChirpStack Network Server connection and vice versa, in the network server details.

\section{Gateway-profiles}
The Gateway-profiles contain the added gateway profiles.
To create a new one, a user presses the Add button and fills in the mandatory fields of the enabled channels, which can be separated with commas, and by choosing a network server from a drop-down list.
The gateway profiles are not used in the courses the school use the ChirpStack with, and the feature is also to be removed in the next major release of ChirpStack, which is yet to have a release date.

\section{Organizations}
The Organizations layout consists of the inserted organizations to the ChirpStack.
The list view provides information of the organizations' name, display name and its possibility of having gateways.
Those details are also to be filled in if the user adds a new organization by clicking the Create button.

\section{All users}
All users consists of the users added to ChirpStack.
The list view shows the username, active status and admin privileges of the users.
A new user can be added by clicking the Create button and providing the mandatory fields of username, e-mail address and password.
Additional details are the optional note field and permission tick boxes which allow selecting if the user is active and giving global admin privileges.

When the user selects an organization from the drop-down list of the second section, the details of the organization are used with all the buttons below to check and modify the information on the layouts the buttons navigate to.

\section{Org. settings}
Org. settings is for updating the settings of the selected organization of the drop-down list.
Adjustable information is the same one that is used, when a new organization is created on the previously mentioned Organizations layout.

\section{Org. users}
Org. users is a layout where the users assigned to the selected organizations can be seen and new ones can be added.
The list view shows the ID of the user, their username and if the user has admin privilege on that organization.
The username is a hyperlink that directs to a layout where the username is seen and where the admin privilege of the organization can be granted or removed.
The layout provides three buttons called Delete, Update user, and Goto user.
The Delete button removes a user from the organization.
The Update user button is used if changes to privilege have been made.
The Goto user button directs to a similar layout to what the All Users layout had for creating a new user with fields populated with the details given upon user creation.
\clearpage

\section{Service-profiles}
The Service-profiles layout provides a list of service profile names that are appended to the organization.
The name is a hyperlink that leads to a page where details can be updated.
When a new service profile is added from the Create button a new layout opens.
Service profile creation requires a service profile name and a network server from a drop-down list of servers that are part of the organization.
Additional details contain tickboxes to allow adding gateway meta data and to enable network geolocation.
In addition three fields can be ticked to give the Device-status request frequency, Minimum allowed data-rate and Maximum allowed data-rate.

\section{Device-profiles}
The Device-profiles is otherwise similar to the Service-profiles layout with the difference of the form of adding or modifying a device profile.
That form includes five tabs: General, Join (\gls{otaa} / \gls{abp}), Class-B, Class-C and Codec.
The General tab includes two mandatory text fields called Device-profile name and Max \gls{eirp} and an optional text field called Geolocation buffer \gls{ttl} (seconds).
The view also has three mandatory drop-down list fields where the Network-server, \gls{lorawan} \gls{mac} version and \gls{lorawan} Regional Parameters revision need to be selected.
The form also contains a field where the Geolocation buffer \gls{ttl} can be selected.
A device profile can already be created after filling the general information, but if other features are to be added, all mandatory fields from that layout of the tab need to be filled in.
The Join (\gls{otaa} / \gls{abp}) tab has a tick box that the user can select if the device supports \gls{otaa}.
The rest of the fields are mandatory and consist of four incremental input fields called RX1 delay, RX1 data-rate offset, RX2 data-rate, and RX2 channel frequency (Hz).
The mandatory regular text field is called Factory-preset frequencies (Hz).
The Class-B tab has one tick box which determines if the device supports Class-B.
The Class-C tab  has the same tick box, but for Class-C device support and also a mandatory incremental input field for Class-C confirmed downlink timeout.
The Codec tab provides an optional drop-down list for selecting the Payload codec.

\section{Gateways}
The Gateways layout provides a list view of the gateways that are added to the selected organization. 
It also delivers a Create button to add new gateways.
The list view has three items it shows of each gateway, the Name, Gateway ID, and Gateway activity (30d).
The name is a hyperlink from where the gateway details can be accessed and modified.
The Gateway ID gives the user a fast way to check the ID of the gateway. 
Gate activity (30d) provides a bar chart figure to have a quick overview of the frames that are sent and received in the last 30 days.
More specified information of the frames is found in the gateway details, where both communication ways have their own charts.

\section{Multicast-groups}
The Multicast-groups layout provides a list view of multicast-groups that are generated for the organization.
This feature is not used by the school.

\section{Applications}
The Applications layout is where the user ends up when they log in.
The page includes a list view of the added applications on the organization that is chosen on the sidebar.
Each item of the list is shown with its ID, name, service-profile and description.
The Device name is a hyperlink that leads to the list of devices that are added to the application, and the service-profile is a hyperlink that leads to the modifying page of the service-profile.
From the Applications layout the user can either choose an existing application or create new one.
Normally when a new course is organized, the educator creates a new application for that class, and all the related and used \gls{lora} end devices are added to that.

In order to create a new application, the user (educator) clicks the Create button on the application layout and a new view opens with a form that has three mandatory fields to fill, as shown in Figure~\ref{fig:ChirpStack_new_application}.
These fields are the application name, application description, and Service-profile of the application.
The first two are text input fields and the third one is a drop-down list.
After filling in the required information that the user clicks, the Create application button and the application are generated.
The page then proceeds back to the list of applications.

\begin{figure}[ht]
  \centering
  {\includegraphics[width=\textwidth]{illustration/ChirpStack_new_application.png}}
  \caption{The new application creation form.}
  \label{fig:ChirpStack_new_application}
\end{figure}

When a desired application is opened, a new layout with a list of current added devices for that application is shown.
When a new application is opened, the device list is empty but visible.
That layout is opened to the Devices heading which is one of the four available ones that an application contains.
The other headers, which can be selected, are Application configuration, Integrations and \gls{fuota}.
Application configuration contains the current information of the application name and description, which both can be updated from there if needed.
Integrations shows a list of added integrations and new ones can be included with a list of available ones on a drop-down list if Create button is pressed.
The \gls{fuota} layout provides information about the Firmware Update Over the Air.
This feature is not used in the classes the school provides and, therefore, was not included in the project. 

The device list shows each linked device provided with the information of when it is last seen, what are the device name and the device \gls{eui}.
It also shows the link margin of the device and the latest information about the battery of each device as seen in Figure~\ref{fig:ChirpStack_application}.
Certain information is required for a user to add devices to the application.

\begin{figure}[ht]
  \centering
  {\includegraphics[width=\textwidth]{illustration/ChirpStack_application.png}}
  \caption{List of devices attached on an application on a ChirpStack network server.}
  \label{fig:ChirpStack_application}
\end{figure}

\subsubsection{Device}
When Create button is pressed in the application layout the \gls{ui} opens a create form to connect the new device.
This form has three sections, general, variables, and tags as seen in Figure~\ref{fig:ChirpStack_new_device}.
For this project only the general section needs to be focused on.
The General section has four mandatory fields.
In order to proceed to the next step, device name, device description, device \gls{eui} and device-profile must be filled in.
The Device name and description are decided by the educator.
Device name can be used on multiple applications, but only once in one application.
Device \gls{eui} can be found when the device is turned on and a Python script main.py is run.
The \gls{eui} is unique and therefore there can only be one device for each \gls{eui}.
Lastly, the device-profile is selected from a drop-down list.
The layout of ChirpStack also provides the possibility to generate a random ID for the device \gls{eui}, to toggle the byte order of it, and to check a box that disables the frame-counter validation.
These features are not used in the project.

\clearpage

\begin{figure}[ht]
  \centering
  {\includegraphics[width=\textwidth]{illustration/ChirpStack_new_device.png}}
  \caption{General information form for adding a new device to the application.}
  \label{fig:ChirpStack_new_device}
\end{figure}

After the required information is given, the user clicks on Create device button.
This process creates and adds the device to the application and the page uploads a layout of the Keys (\gls{otaa}) information of the device as seen in Figure~\ref{fig:ChirpStack_new_device_2}.
The mandatory application key can also be accessed from the main.py file.
After key information is given, the user clicks the Set device keys button and the page uploads back to the opened application layout.
From the list view, the user can find the just added device.

The details tab is opened by clicking on the hyperlink on the device name.
The opened device has seven tab headers in total that are the current details, keys (\gls{otaa}), activation, device data, \gls{lorawan} frames, and firmware.
Next the information about the content of those tabs is explained.

The Details layout has four text boxes called Details, Status, Enqueue downlink payload, and Downlink queue.
The Details and Status boxes are visible all the time after a device is added to the application but the Enqueue downlink payload and Downlink queue only become visible after the device has been connected to the server for the first time.

Details shows the name and description of the opened device and a hyperlink to the Device-profile that has been selected for it.
The hyperlink leads to the general configuration page of that device-profile.
The Status textbox has a Last seen at field that provides the information of the latest time that the device has been connected on the application.
Enqueue downlink payload feature provides possibility to transmit data to a queue from where it is sent to the \gls{lora} device.
For this to happen, the user needs to give a port number and a Base64 encoded string or a JSON object and the user can also tick a box if the downlink is confirmed.
The Downlink queue shows  a list of the downlinks that are on the queue while providing the main information about them in four columns.
The downlink features are not used in the courses currently.

The Configuration layout is the same one as when the user had clicked the Create button on the Devices tab of the application to create a new device as seen previously in Figure~\ref{fig:ChirpStack_new_device}.
From there the user can modify the general details of the device if wanted.
If this is done, the new information is saved by pressing the Update device button.
When the details are updated, the page layout goes to the Details tab where the modified data can be seen.

The Keys (\gls{otaa}) layout is the tab, which was opened when the first step of the creation was done.
The text box has two fields, Application key and Gen application key.
The Application key is a mandatory field and Gen Application key is optional.
The Application key can be accessed when a Python main.py file from the device is run.
The Application key can either be unique or the same with another device.
What matters is that the key is defined to be the same in the Python script of the \gls{lora} end device, which is used when the device is booted, and in the device details when the device is added.
The Gen Application key is only needed when the device implements remote multicast setup specification or \gls{fuota}.
As neither of those features are used in the implementation, that field is left empty.

The Activation layout provides a textbox with five fields when the device has been connected to the server / activated.
The First field shows the Device address which is specified when the device has joined a network.
The Network session key and the Application session key are masked.
The concealed information can be seen by pressing the key icons next to the next fields.
The other two fields show the Uplink frame-counter and Downlink frame-counter information.

The Device data layout sends uplink information from the device when it is connected in five second intervals.
If it is not activated, the box instead shows a text, "This device has not been (yet) activated".

\clearpage

\begin{figure}[ht]
  \centering
  {\includegraphics[width=\textwidth]{illustration/ChirpStack_new_device_2.png}}
  \caption{Keys (\gls{otaa}) information form for adding a new device to the application.}
  \label{fig:ChirpStack_new_device_2}
\end{figure}

The device can only be added once to the ChirpStack server, and if it is tried to add another time to either the same application it already exists on or to a new one, the process fails with an error message.

%\cite{ChirpStack:devices}

\clearpage %force the next chapter to start on a new page. Keep that as the last line of your chapter!

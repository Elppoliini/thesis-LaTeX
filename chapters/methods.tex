% Material and Methods

\chapter{Material and Methods} \label{ch:mat_met}
\section{LoRaWAN}
LoRaWAN is A networking protocol that is based on \gls{lora} and it is maintained by the LoRa Alliance (LINK). It is designed to operate with low-power, battery operated devices. LoRaWAN enables long range communication between devices for up to 15 kilometers which is useful in ares where the devices are spread out. (SOURCE https://lora-developers.semtech.com/documentation/tech-papers-and-guides/lora-and-lorawan/). The long-range enables cost-effective deployments by allowing to cover large areas with a minimal number of gateways. The LoRaWAN signals have good penetration capabilities approving them travel through obstacles like walls and buildings. Many \gls{iot} devices transmit small amounts of data at irregular intervals and using LoRaWan that has the ability to transmit data at low rates is therefore a well suitable option. The networks can scale to support a large number of devices which is important for \gls{iot} implementations that gather data from a diverse range of sensors. These features make LoRaWAN well-suited for \gls{iot} deployments. It makes a wireless connection between the end devices and the gateway.
\section{LORIX One}
LORIX One is a LoraWan Gateway that is used to connect the end devices to the internet
\section{Chirpstack}
    Chirpstack is a LoRaWAN network server. It is an open-source platform, which means the source code is freely accessible, enabling contributions and modifications from individuals. It allows anyone to engage in its development. LORIX One is connected to it and this way the end devices can be added to the server.
\section{RobotFramework}



Check Final Year Project Guide for the content of Material and Methods chapter.

\clearpage %force the next chapter to start on a new page. Keep that as the last line of your chapter!

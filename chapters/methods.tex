% Material and Methods

\chapter{Material and Methods} \label{ch:mat_met}
\section{LoRaWAN}
LoRaWAN is A networking protocol that is based on \gls{lora} and maintained by the LoRa Alliance.
Specifically designed for low-power, battery-operated devices, LoRaWAN facilitates long-range communication between devices, extending up to 15 kilometers.
This extended range proves invaluable in areas where devices are widely distributed, optimizing connectivity.
The long-range capability contributes to cost-effective deployments, enabling the coverage of extensive areas with a minimal number of gateways.
LoRaWAN signals exhibit excellent penetration capabilities, allowing them to traverse obstacles like walls and buildings, enhancing communication reliability.
Many \gls{iot} devices operate by transmitting small amounts of data at irregular intervals.
LoRaWAN, with its capacity to transmit data at low rates, emerges as an optimal solution for such occasional data transmissions.
This low data rate feature not only conserves energy in battery-operated devices but also ensures efficient use of network resources.
LoRaWAN networks boast scalability, accommodating a substantial number of devices.
This scalability proves vital for \gls{iot} implementations that collect data from a diverse array of sensors.
The adaptability to scale makes LoRaWAN well-suited for the dynamic and evolving nature of IoT deployments.
In this project, LoRaWAN technology plays a significant role in wireless connectivity.
The implementation involves LoRa end devices utilizing Over-The-Air Activation (OTAA) to establish connections with the LORIX One gateway.
Functioning as a bridge, the gateway connects these devices to the Chirpstack LoRaWAN network server.
At the beginning of each course, the teacher initiates the creation of a new application on the server and the addition of devices to that application.
This manual process, characterized by repetition, is targeted for automation to minimize the time commitment from educators.
The integration of LoRaWAN not only optimizes connectivity in widely distributed educational settings but also capitalizes on its long-range capabilities, low data rate support, and scalability, making it an efficient solution for IoT applications within the project.
\cite{lora-developer-portal:about}
CHECK  (\cite{}) https://lora-developers.semtech.com/documentation/tech-papers-and-guides/lora-and-lorawan/ FOR POSSIBLE IDEAS FOR FIGURES RELATED TO LORAWAN

\section{LORIX One}
LORIX One is a LoRaWAN gateway developed by Wifx.
The operating system that LORIX One uses is called LORIX OS and it is created particularly for LoRaWAN gateways.
In this project LORIX One is used to transmit data between the \gls{iot} devices and the Chirpstack Network Server.
\section{Chirpstack}
Chirpstack is a LoRaWAN network server. It is an open-source platform, which means the source code is freely accessible, enabling contributions and modifications from individuals. It allows anyone to engage in its development. LORIX One is connected to it and this way the end devices can be added to the server.
\section{Robot Framework}
Robot Framework is an open-source framework supported by Robot Framework Foundation and can be used for test automation and \gls{rpa}.
Robot Framework uses keywords that are in a human-readable format making its syntax easily accessible for technical and non-technical users.
The framework supports both data-driven and keyword-driven approaches for the implementations.
In data-driven approach the logic is separated from the data which allows to use the logic easily with multiple different inputs or data sets and keyword-driven approach focuses on creating keywords that give summary of the actions that the robot performs and which makes the maintainability, and readability more straightforward.
Robot Framework contains standard libraries from which BuiltIn is the most used one and imported automatically.
The features can be extended by libraries that can be implemented in multiple different programming languages.
The project is implemented by using the Robot Framework for its broad possibilities to be developed and maintained further if needed, the possibilities of cross platform development and for its rich ecosystem, that eases the previous reasons by providing active support from the community, forums and other resources.
The following subsections will provide explanations of what libraries are used in the implementation of this project.

\subsection{Standard Library}
As mentioned in the beginning of this section, Robot Framework contains Standard Library that consist of ten libraries that can be imported to the projects with the exception of BuiltIn library that is imported by default.
These libraries include a wide spectrum of the most common tasks and verifications related to things such as date and time manipulations, collection handling, user inputs, operation system related tasks, process management and many others.
This project contains used two of those libraries, the BuiltIn library and the Dialogs libary.
Further information about them is told in the next subsections.
\subsubsection{BuiltIn}
As mentioned previously BuiltIn is the only Standard Library in Robot Framework that is imported automatically. It provides keywords that are the most generic ones and therefore needed often.
\subsubsection{Dialogs}
\subsection{Browser Library}
\subsection{RPA Framework}
\subsubsection{RPA.Selenium.Browser}
\subsubsection{RPA.Excel.Files}

\cite{robot-framework:standard-library}
\clearpage %force the next chapter to start on a new page. Keep that as the last line of your chapter!

% Material and Methods

\chapter{Material and Methods} \label{ch:mat_met}
\section{LoRaWAN}
LoRaWAN is A networking protocol that is based on \gls{lora} and maintained by the LoRa Alliance.
Specifically designed for low-power, battery-operated devices, LoRaWAN facilitates long-range communication between devices, extending up to 15 kilometers.
This extended range proves invaluable in areas where devices are widely distributed, optimizing connectivity.
The long-range capability contributes to cost-effective deployments, enabling the coverage of extensive areas with a minimal number of gateways.
LoRaWAN signals exhibit excellent penetration capabilities, allowing them to traverse obstacles like walls and buildings, enhancing communication reliability.
Many \gls{iot} devices operate by transmitting small amounts of data at irregular intervals.
LoRaWAN, with its capacity to transmit data at low rates, emerges as an optimal solution for such occasional data transmissions.
This low data rate feature not only conserves energy in battery-operated devices but also ensures efficient use of network resources.
LoRaWAN networks boast scalability, accommodating a substantial number of devices.
This scalability proves vital for \gls{iot} implementations that collect data from a diverse array of sensors.
The adaptability to scale makes LoRaWAN well-suited for the dynamic and evolving nature of IoT deployments.
In this project, LoRaWAN technology plays a significant role in wireless connectivity.
The implementation involves LoRa end devices utilizing Over-The-Air Activation (OTAA) to establish connections with the LORIX One gateway.
Functioning as a bridge, the gateway connects these devices to the Chirpstack LoRaWAN network server.
At the beginning of each course, the teacher initiates the creation of a new application on the server and the addition of devices to that application.
This manual process, characterized by repetition, is targeted for automation to minimize the time commitment from educators.
The integration of LoRaWAN not only optimizes connectivity in widely distributed educational settings but also capitalizes on its long-range capabilities, low data rate support, and scalability, making it an efficient solution for IoT applications within the project.
\cite{lora-developer-portal:about}
CHECK  (\cite{}) https://lora-developers.semtech.com/documentation/tech-papers-and-guides/lora-and-lorawan/ FOR POSSIBLE IDEAS FOR FIGURES RELATED TO LORAWAN

\section{LORIX One}
LORIX One is a LoRaWAN gateway developed by Wifx.
The operating system that LORIX One uses is called LORIX OS and it is created particularly for LoRaWAN gateways.
In this project LORIX One is used to transmit data between the \gls{iot} devices and the Chirpstack Network Server.

\section{Chirpstack}
Chirpstack is a LoRaWAN network server. It is an open-source platform, which means the source code is freely accessible, enabling contributions and modifications from individuals. It allows anyone to engage in its development. LORIX One is connected to it and this way the end devices can be added to the server.

\section{Robot Framework}
Robot Framework is an open-source framework supported by Robot Framework Foundation and can be used for test automation and \gls{rpa}.
Robot Framework uses keywords that are in a human-readable format making its syntax easily accessible for technical and non-technical users.
The framework supports both data-driven and keyword-driven approaches for the implementations.
In data-driven approach the logic is separated from the data which allows to use the logic easily with multiple different inputs or data sets and keyword-driven approach focuses on creating keywords that give summary of the actions that the robot performs and which makes the maintainability, and readability more straightforward.
Robot Framework contains standard libraries from which BuiltIn is the most used one and imported automatically.
The features can be extended by libraries that can be implemented in multiple different programming languages.

This section will provide understanding of what test suites and libraries are and and how they can be used in the implementation of this project.

\subsection{Test suites}
A test case file that contains the automation tasks creates a test suite. Usually a test suite consists of ten or fewer tasks and the directory can have multiple test suites.
Automation of different processes can include similar pre-steps or afterwork that needs to be done every time a test suite is executed.
To make things more efficient a user can create Suite setup and teardown keywords that assist by doing the repeatedly needed steps.
These keywords have also the ability to accept arguments as needed.
Suite setup and teardown are declared in the suite initialization file.

A Suite setup is a keyword that is executed before any tasks are run.
If the setup fails, all the test suites that utilize it are directly assigned to fail status without being executed.
Suite setup offers possibilities to check preconditions that must be fulfilled before the automation tasks are being performed.

Similarly to a Suite setup, a Suite teardown is performed after the automation tasks are executed.
It is usually performed to clean up, and with a difference to a suite, if the suite setup fails the teardown is still being executed.
If the suite teardown fails, the test suites are marked as fail status regardless of their original status after execution.

\subsection{Standard Library}
As mentioned in the beginning of this section, Robot Framework contains Standard Library that consist of ten libraries that can be imported to the projects with the exception of BuiltIn library that is imported by default.
These libraries include a wide spectrum of the most common tasks and verifications related to things such as date and time manipulations, collection handling, user inputs, operation system related tasks, process management and many others.
This project utilizes two of those libraries, the BuiltIn library and the Dialogs library.
Further information about them is explained in the next subsections.
\subsubsection{BuiltIn}
As mentioned previously BuiltIn is the only Standard Library in Robot Framework that is imported automatically.
It provides keywords that are the most generic ones and therefore needed often.
\subsubsection{Dialogs}
Dialogs is part of the Standard Library and provides keywords to interact with the user. The keywords can be used to things such as to take user input or pause the execution until the user 
\subsection{Browser Library}
Browser library is a Robot Framework library that delivers keywords for browser automation.
It uses Playwright framework to automate Chromium, Firefox and webkit.
The library offers a wide range of keywords, including getters for fetching information from various elements, classes, texts, and \gls{url}s.
Additionally, it includes keywords for waiting until certain elements are loaded or become visible before proceeding with the execution of subsequent keywords
\subsection{\gls{rpa} Framework}
Robocorp sponsors a  project of \gls{rpa} Framework, which is a collection of tools and libraries that can be used for \gls{rpa}.
The libraries are released under an open-source license.
Both the libraries and tools can be used with Robot Framework or Python.
This project uses one of the libraries from the collection, the RPA.Excel.Files.
\subsubsection{RPA.Excel.Files}
Robot framework features a library called RPA.Excel.Files which provides a large variation of keywords used to read and write data to Excel files.
The library enables the use of those files without the necessity of opening the Excel application.
\cite{robot-framework:standard-library}
\cite{robot_framework:builtin_library}
\cite{robot_framework:dialogs_library}
\cite{robot_framework:browser_library}
\cite{rpa_framework:excel_files}
\clearpage %force the next chapter to start on a new page. Keep that as the last line of your chapter!

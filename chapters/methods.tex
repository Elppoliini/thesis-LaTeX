% Material and Methods

\chapter{Material and Methods} \label{ch:mat_met}
\section{LoRaWAN}
LoRaWAN is A networking protocol that is based on \gls{lora} and maintained by the LoRa Alliance (LINK).
Specifically designed for low-power, battery-operated devices, LoRaWAN facilitates long-range communication between devices, extending up to 15 kilometers.(SOURCE https://lora-developers.semtech.com/documentation/tech-papers-and-guides/lora-and-lorawan/)
This extended range proves invaluable in areas where devices are widely distributed, optimizing connectivity.
The long-range capability contributes to cost-effective deployments, enabling the coverage of extensive areas with a minimal number of gateways.
Notably, LoRaWAN signals exhibit excellent penetration capabilities, allowing them to traverse obstacles like walls and buildings, enhancing communication reliability.
Many \gls{iot} devices operate by transmitting small amounts of data at irregular intervals.
LoRaWAN, with its capacity to transmit data at low rates, emerges as an optimal solution for such occasional data transmissions.
This low data rate feature not only conserves energy in battery-operated devices but also ensures efficient use of network resources.
LoRaWAN networks boast scalability, accommodating a substantial number of devices.
This scalability proves vital for \gls{iot} implementations that collect data from a diverse array of sensors.
The adaptability to scale makes LoRaWAN well-suited for the dynamic and evolving nature of IoT deployments.
In this project, LoRaWAN technology plays a significant role in wireless connectivity.
The implementation involves LoRa end devices utilizing Over-The-Air Activation (OTAA) to establish connections with the LORIX One gateway.
Functioning as a bridge, the gateway connects these devices to the Chirpstack LoRaWAN network server.
At the beginning of each course, the teacher initiates the creation of a new application on the server and the addition of devices to that application.
This manual process, characterized by repetition, is targeted for automation to minimize the time commitment from educators.
The integration of LoRaWAN not only optimizes connectivity in widely distributed educational settings but also capitalizes on its long-range capabilities, low data rate support, and scalability, making it an efficient solution for IoT applications within the project.

CHECK  (\cite{}) https://lora-developers.semtech.com/documentation/tech-papers-and-guides/lora-and-lorawan/ FOR POSSIBLE IDEAS FOR FIGURES RELATED TO LORAWAN

\section{LORIX One}
LORIX One is a LoraWan Gateway that is used to connect the end devices to the internet
\section{Chirpstack}
    Chirpstack is a LoRaWAN network server. It is an open-source platform, which means the source code is freely accessible, enabling contributions and modifications from individuals. It allows anyone to engage in its development. LORIX One is connected to it and this way the end devices can be added to the server.
\section{RobotFramework}



Check Final Year Project Guide for the content of Material and Methods chapter.

\clearpage %force the next chapter to start on a new page. Keep that as the last line of your chapter!

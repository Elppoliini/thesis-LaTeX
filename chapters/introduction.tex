% Introduction

\chapter{Introduction}

In the swiftly moving field of technology, automation is increasing widely, offering advantages such as enhanced reliability, efficiency, and availability.
Two important thing in this field are  \gls{iot} and \gls{rpa}. Specifically, \gls{rpa} is a significant tool for making complicated tasks easier by efficiently handling repetitive tasks, while \gls{iot} enhances the overall environment by connecting devices and enabling smart functionalities. 

This thesis is presented to Metropolia University of Applied Sciences to benefit their courses that are related with the wireless technologies. In these courses the students utilize \gls{lora} end devices that have \gls{otaa} to establish connection to the LORIX One gateway. The gateway acts as the bridge, connecting the devices to the Chirpstack \gls{lorawan} network server. In the beginning of a course the teacher needs to create a new application to the server and add all the devices to that application. This includes a lot of repeating work that can be automated. The implementation seeks to minimize the time commitment from the teachers at the onset of each course by providing automation for data transcription. The objective is to enhance overall efficiency, enabling lecturers to allocate more attention to instructional activities.

This thesis contains 6 chapters.
Chapter~\ref{ch:theor_backgr} describes the theoretical foundation of the project, exploring potential alternatives to the chosen implementation method.
Chapter~\ref{ch:mat_met} presents detailed information on \gls{lorawan}, LORIX One, ChirpStack, and the Robot Framework and provides a comprehensive understanding of these key components.
Chapter~\ref{ch:project_spec} offers a detailed description of the project's specifications, offering a broad overview of the scope and requirements. 
Chapter~\ref{ch:impl} focuses on the practical aspects of the implementation process, offering a step-by-step breakdown.
Chapter~\ref{ch:conc} summarizes the findings and insights obtained throughout the entire thesis.


\clearpage %force the next chapter to start on a new page. Keep that as the last line of your chapter!

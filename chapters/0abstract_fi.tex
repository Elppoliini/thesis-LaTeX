% Abstract in Finnish
%Most probably, you only need to change the text of the abstract. Everything else comes from chapter/0info.tex
%If you do not have any appendix, you may delete \total{chapter} and replace with 0

\thispagestyle{tiivis}
\begin{otherlanguage}{finnish}
{\renewcommand{\arraystretch}{2}%
\begin{tabular}{ | p{4,7cm} | p{10,3cm} |}
  \hline
  Tekijä(t) \newline
  Otsikko \newline\newline 
  Sivumäärä \newline
  Aika
  & 
  \makeatletter
  \@author \newline 
  \otsikko \newline\newline %! if very long title over 2 lines, remove one \newline
  \makeatother
  \pageref*{LastPage} sivua + \total{chapter} liitettä \newline %! if no appendices, risk to count total of chapter :D
  \pvm		
  \\ \hline
  Tutkinto & \tutkinto
  \\ \hline
  Tutkinto-ohjelma & \kohjelma
  \\ \hline
  Ammatillinen pääaine & \suuntautumis
  \\ \hline
  Ohjaaja(t) & \ohjaajat
  \\ \hline
  \multicolumn{2}{|p{15cm}|}{\vspace{-22pt}
  Robot Framework on avainsanaohjattu avoimen lähdekoodin automaatiokehys, jota voidaan käyttää testiautomaatioon ja  ohjelmistorobotiikkaan. Se on laajennettavissa useilla kirjastoilla, jotka voidaan totetuttaa eri ohjelmointikielillä ja sen syntaksi käyttää luettavia avainsanoja. \newline

  Tämä opinnäytetyö perustuu Robot Framework automaatio- toteutukseen \gls{lora} laitteille, jotka käyttävät \gls{otaa} metodia palvelimen kanssa kommunikointiin ja opinnäytetyö suoritettiin Metropolia Ammattikorkeakoululle. Työn materiaali kerättiin tutkimalla verkkomateriaalia Robot Frameworkille tarjolla olevista kirjastoista, perehtymällä yksityiskohtaisempaan materiaaliin liittyen ChirpStack verkkopalvelimeen ja LORIX One reitittimeen, sekä tutustumalla Metropoliassa aiemmin opiskelleen henkilön opinnäytetyöhön, jossa \gls{lora} päätelaite oli kehitetty. Opinnäytetyön toteutusta testattiin kehityksen ohessa ajamalla ohjelmakoodia ja varmistamalla että yhteydet laitteen, palvelimen ja reitittimen välillä toimivat asianmukaisesti. \newline

  Toteutus tarjoaa ohjelmakoodin, joka lisää \gls{lora} päätelaitteen äskettäin luotuun sovellukseen ChirpStack verkkopalvelimeen, mihin laitteilla on yhteys   \gls{otaa} metodin avulla käyttämällä LORIX One reititintä \gls{lorawan} protokollalla. \newline

  Tämän opinnäytetyön lopputulos auttaa Metropolia Ammattikorkeakoulua poistamalla toistuvia tehtäviä, joita opettajien on tarvinnut tehdä langattomien teknologioiden kurssien alussa ja siten tarjoamalla enemmän aikaa kohdennettavaksi opetustoimintaan. \newline
  } \\[14cm] \hline
  Avainsanat & \avainsanat
  \\ \hline
\end{tabular}
}
\end{otherlanguage}
\clearpage


% Project Specifications

\chapter{Project Specifications} \label{ch:project_spec}
This chapter explains further information about the starting point and the desired outcome of this project.
The project was planned and developed for Metropolia University of Applied Sciences to automate processes used in wireless communication related courses in the school.
The chapter 2 explained how the manual process of this projects implementation works giving good understanding of the starting point.
This also gave all the physical tools required, meaning the \gls{lora} devices that the students use and that are to be added to the application on ChirpStack server.
The Chirpstack \gls{os} which is installed on a micro\gls{sd} card is to be run with the LORIX One gateway.
As there is no current plan to update any of them from the school's side of view they are also not to be replaced for this project.

The project's automation is implemented by using the Robot Framework for its broad possibilities to be developed and maintained further if needed, the possibilities of cross platform development and for its rich ecosystem, that eases the previous reasons by providing active support from the community, forums and other resources.

Chapter 3 introduced what all the methods that have been chosen for this projects are.
This chapter walks through why those methods were chosen and how they are going to be used.

\todo{They have 
a clear idea of what needs to be done and what tools/systems/devices etc
student describes the project specifications in as much detail as 
possible and explains the reasoning behind them}

\clearpage %force the next chapter to start on a new page. Keep that as the last line of your chapter!
